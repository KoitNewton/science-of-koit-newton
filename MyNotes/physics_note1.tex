\documentclass[12pt, a4paper, oneside]{ctexbook}
\usepackage{amsmath, amsthm, amssymb, bm, graphicx, hyperref, mathrsfs}

\title{{\Huge{\textbf{ミクロ世界の導論}}}\\——Quantum Mechanics}
\author{Koit Newton}
\date{\today}
\linespread{1.5}
\newtheorem{theorem}{定理}[section]
\newtheorem{definition}[theorem]{定义}
\newtheorem{lemma}[theorem]{引理}
\newtheorem{corollary}[theorem]{推论}
\newtheorem{example}[theorem]{例}
\newtheorem{proposition}[theorem]{命题}

\begin{document}                                                            

\maketitle

\pagenumbering{roman}
\setcounter{page}{1}

\begin{center}
    \Huge\textbf{前言}
\end{center}~\

这是笔记的前言部分. 
~\\
\begin{flushright}
    \begin{tabular}{c}
        Koit Newton\\
        \today
    \end{tabular}
\end{flushright}

\newpage
\pagenumbering{Roman}
\setcounter{page}{1}
\tableofcontents
\newpage
\setcounter{page}{1}
\pagenumbering{arabic}

\chapter{经典物理基础}
系统总能量算符: $\hat{E}=i \hbar \frac{\partial}{\partial t} ;$ 哈密顿量算符: $\hat{H}=\hat{T}+\hat{V}$ 在球坐标中, 角动量算符的各个分量分别为
$$
\begin{gathered}
\hat{L}_x=\hat{y} \hat{\boldsymbol{P}}_z-\hat{z} \hat{P}_y=-i \hbar\left(y \frac{\partial}{\partial z}-z \frac{\partial}{\partial y}\right)=i \hbar\left(\sin \varphi \frac{\partial}{\partial \theta}+\cot \theta \cos \varphi \frac{\partial}{\partial \varphi}\right) \\
\hat{L}_y=-i \hbar\left(\cos \varphi \frac{\partial}{\partial \theta}-\cot \theta \sin \varphi \frac{\partial}{\partial \varphi}\right) \\
\hat{L}_z=-i \hbar \frac{\partial}{\partial \varphi} \\
\hat{L}^2= - \hbar^2 \left( \frac{1}{\sin \theta} \frac{\partial}{\partial \theta} \sin \theta \frac{\partial}{\partial \theta} + \frac{1}{\sin^2 \theta} \frac{\partial^2}{\partial \phi^2} \right)
\end{gathered}
$$
相应的动能算符则为
$$
\hat{T}=-\frac{\hbar^2}{2 m} \frac{1}{r} \frac{\partial^2}{\partial r^2} r+\frac{\hat{L}^2}{2 m r^2}
$$


\section{小节标题}

这是笔记的正文部分. 

\end{document}